\documentclass[aps,ams,prl,twocolumn,superscriptaddress]{revtex4-1}

\usepackage{graphicx}  % this is the up-to-date package for all figures
\usepackage{amssymb}   % for math
\usepackage{verbatim}  % for the comment environment
\usepackage{color}
\usepackage[hidelinks]{hyperref}
\usepackage{listings}	% for code snippets
\usepackage{gensymb}

% Settings for code appendices
\definecolor{dkgreen}{rgb}{0,0.6,0}
\definecolor{gray}{rgb}{0.5,0.5,0.5}
\definecolor{mauve}{rgb}{0.58,0,0.82}

\lstset{frame=tb,
  language=C++,
  aboveskip=3mm,
  belowskip=3mm,
  showstringspaces=false,
  columns=flexible,
  basicstyle={\small\ttfamily},
  numbers=none,
  numberstyle=\tiny\color{gray},
  keywordstyle=\color{blue},
  commentstyle=\color{dkgreen},
  stringstyle=\color{mauve},
  breaklines=true,
  breakatwhitespace=true,
  tabsize=3
}

%\bibliographystyle{apsrev}

% these are some custom control of the page size and margins
% \topmargin= 0.2in  % these 1st two may be needed for some computers
\textheight=9in
\textwidth=6.5in
% these next two lines give us centered text
\oddsidemargin=0cm
\evensidemargin=0cm

\begin{document}

\pagestyle{headings}

\title{Polar Stereographic\\ Heightfield Intersection}

\author{Lukas R. Hertrich}
\affiliation{Student Researcher at University of Hawaii at Manoa}

\begin{abstract}
In order to simulate neutrino and RF traversal of the Antarctic firn, raycasting may be used. This is accomplished by mapping bedrock elevation and ice sheet thickness
data, provided by projects such as BEDMAP 2, to the WGS84 ellipsoid.
\end{abstract}

\maketitle

\section{Data Format}
Antarctic elevation and ice sheet data are provided by BEDMAP 2 in Antarctic polar stereographic projection.
The projection plane is located at 71$\degree$ south latitude and rays are cast down from the north pole to intersect the projection plane and ellipsoid.
The angle is in ellipsoidal coordinates determined by WGS84.
\section{Raycast}
The 3D raycast is parameterized with an initial position $\vec{r_0}$, a direction $\hat{d}$, and a parameter $\tau$, yielding
\begin{equation}\label{eqn:r_of_tau}
 \vec{r}(\tau) = \vec{r_0} + \tau \hat{d}
\end{equation}
This ray traverses flat, unmodified 3D cartesian space, meaning that distance traversed by the ray is equal to $\tau$.
The same ray may be parameterized in terms of normalized cartesian coordinates,
meaning that the $\vec{r_0}$ terms $r_{0x}$ and $r_{0y}$ are divided by the equatorial radius $R_{eq}$ of the WGS84 ellipsoid,
and the $r_{0z}$ term is divided by the polar radius $R_{po}$.
This yields a magnitude of radius relative to the radius of the ellipsoid in the given direction, and a ray that intersects the ellipsoid's surface
where $r_{el}(\tau) = 1$.
The magnitude of $r_{el}$ may be determined as follows:
\begin{eqnarray*}\label{eqn:r_el}
 r_{el} & = & \left[\frac{r_{0x}^2+r_{0y}^2}{R_{eq}^2}+\frac{r_{0z}^2}{R_{po}^2}\right. \\
 & & {} + 2\tau\left(\frac{r_{0x} d_x+r_{0y} d_y}{R_{eq}^2}+\frac{r_{0z} d_z}{R_{po}^2}\right) \\
 & & {} +\left. \tau^2 \left( \frac{d_x^2+d_y^2}{R_{eq}^2} + \frac{d_z^2}{R_{po}^2} \right)\right]^{\frac{1}{2}}
\end{eqnarray}
\section{Heightfield Intersection}
The ray intersects the heightfield when the normalized surface radius $r_{surf}(\tau)$ is equal to the normalized ellipsoidal radius $r_{el}(\tau)$.
Their difference is then zero. Likewise the difference of their squares is zero.
\begin{equation}\label{eqn:intersection}
0 = r_{surf}(\tau) - r_{el}(\tau) 
\end{equation}\\
$r_{el}(\tau)$ may be determined by Eqn. (\ref{eqn:r_el}).
$r_{surf}(\tau)$ is unity added to the quotient of the interpolated height $h(\tau)$ and the unnormalized magnitude of the ellipsoid's radius $R(\tau)$ at the given latitude.
\begin{equation}\label{eqn:r_surf}
 r_{surf}(\tau) = 1 + \frac{h(\tau)}{R(\tau)}
\end{equation}
The interpolated height $h(\tau)$ is determined by generating the vector $\vec{v}_{proj}$ from the north pole of the ellipsoid $(0, 0, R_{po})$
to the point of ray traversal $\vec{r}(\tau)$ from Eqn. (\ref{eqn:r_of_tau}),
then using the point of intersection of this vector with the projection plane to calculate (u,v) coordinates with which to poll the heightfield data.
The point of intersection with the projection plane is found by determining how far the direction of $\vec{v}_{proj}$ must be traversed, in terms of its $\hat{z}$ component,
in order to hit the constant $\hat{z}$ position of the plane $z_{plane}$. This is accomplished by scaling $\vec{v}_{proj}$ by $\alpha_{proj}$ to form $\vec{v'}_{proj}$.
\begin{equation}\label{eqn:v_proj}
 \vec{v'}_{proj} = \alpha_{proj} \vec{v}_{proj}
\end{equation}
The scalar $\alpha_{proj}$ is determined as follows, with $z(\tau)$ being the $z$ component of $r(\tau)$ in Eqn. (\ref{eqn:r_of_tau}):
\begin{equation}\label{eqn:alpha_proj}
 \alpha_{proj} = \frac{-R_{po} - z_{plane}}{-R_{po} + z(\tau)}
\end{equation}
This yields the $(x'_{proj},y'_{proj})$ coordinates that may be converted to $(u,v)$ coordinates to poll and interpolate data.
\begin{equation}\label{eqn:x'y'_proj}
 (x'_{proj},y'_{proj}) = \alpha_{proj} (x_{proj},y_{proj})
\end{equation}
$R(\tau)$ may be determined from the geodetic latitude $\varphi$, measured from the equator.

\begin{equation}\label{eqn:phi}
 \varphi = \arcsin{\frac{z_{el}(\tau)}{r_{el}(\tau)}}
\end{equation}

\begin{equation}\label{eqn:R_of_phi}
 R(\varphi) = \sqrt{\frac{R_{eq}^4\cos^2{\varphi} + R_{po}^4\sin^2{\varphi}}{R_{eq}^2\cos^2{\varphi} + R_{po}^2\sin^2{\varphi}}}
\end{equation}
$R(\tau)$ may be expressed by substituting the expressions for $\cos^2{\varphi}$ and $\sin^2{\varphi}$ into Eqn. (\ref{eqn:R_of_phi}).
\begin{eqnarray*}\label{eqn:cos_sqr_phi}
 \cos^2{\varphi} & = & \frac{x_{el}^2 + y_{el}^2}{r_{el}^2} \\
 && = \frac{(r_{0x} + \tau d_x)^2 + (r_{0y} + \tau d_y)^2}{R_{eq}^2 r_{el}^2}
\end{eqnarray}
\begin{equation}\label{eqn:sin_sqr_phi}
 \sin^2{\varphi} = \frac{z_{el}^2}{r_{el}^2} = \frac{\left(r_{0z} + \tau d_z\right)^2}{R_{po}^2 r_{el}^2}
\end{equation}
Simplifying the substituted terms reduces $R(\tau)$ to:
\begin{eqnarray*}\label{eqn:R_of_tau}
 R(\tau) & = & \left[\frac{R_{eq}^2 \left[ (r_{0x} + \tau d_x)^2 + (r_{0y} + \tau d_y)^2 \right]}{r^2(\tau)}\right. \\
 && {} +\left.\frac{R_{po}^2 (r_{0z} + \tau d_{z})^2}{r^2(\tau)}\right]^{\frac{1}{2}} 
\end{eqnarray}

\section{Interpolation of Height Data}
Precomputing the values of $r_{surf}^2$ at data grid points allows Eqn. (\ref{eqn:intersection}) to be solved analytically for a given data cell.
There is a negligible error introduced by interpolating the square of the surface radius rather than the radius itself.
The hypothetical error introduced by considering the unrealistic situation of adjacent maxima and minima of Antarctic elevation in a data cell yields an error on the order of one meter.
This is well within the range of uncertainty in the BEDMAP 2 measurements themselves, which document uncertainties of as much as 1 km for ice thickness data in certain regions. 

Normalized $(u,v)$ coordinates for interpolation in the range $[0,1]$ are generated using the results of Eqn. (\ref{eqn:x'y'_proj}) and the coordinates at the corners of a given data cell.
\begin{equation}\label{eqn:u}
 u = \frac{x'_{proj} - x_{0,cell}}{x_{1,cell} - x_{0,cell}}
\end{equation}
\begin{equation}\label{eqn:v}
 v = \frac{y'_{proj} - y_{0,cell}}{y_{1,cell} - y_{0,cell}}
\end{equation}
Interpolation of the $r_{surf}^2$ values at the cell corners is accomplished as follows:
\begin{eqnarray*}\label{r_surf_sqr}
 r_{surf}^2 & = & r_{00}^2(1-u)(1-v) \\
 && {} + r_{10}^2(u)(1-v) \\
 && {} + r_{01}^2(1-u)(v) \\
 && {} + r_{11}^2uv
\end{eqnarray}
This results in a quadratic form, in terms of dependence on $\tau$, which can be shown by substituting terms successively:
\begin{eqnarray*}\label{eqn:r_surf_quadratic}
 r_{surf}^2 & = & r_{00}^2 + u(r_{10}^2 - r_{00}^2) + v(r_{01}^2 - r_{00}^2) \\
 && {} + uv(r_{11}^2 - r_{01}^2 - r_{10}^2 - r_{00}^2)
\end{eqnarray}
\begin{eqnarray*}
 r_{surf}^2 & = & r_{00}^2 \\
 && {} + \frac{x'_{proj} - x_{0,cell}}{x_{1,cell} - x_{0,cell}}(r_{10}^2 - r_{00}^2) \\
 && {} + \frac{y'_{proj} - y_{0,cell}}{y_{1,cell} - y_{0,cell}}(r_{01}^2 - r_{00}^2) \\
 && {} + \left(\frac{x'_{proj} - x_{0,cell}}{x_{1,cell} - x_{0,cell}}\right)\left(\frac{y'_{proj} - y_{0,cell}}{y_{1,cell} - y_{0,cell}}\right) \\
 && {} \cdot(r_{11}^2 - r_{01}^2 - r_{10}^2 - r_{00}^2)
\end{eqnarray}
\begin{eqnarray*}
 r_{surf}^2 & = & r_{00}^2 \\
 && {} + \frac{\alpha_{proj} x(\tau) - x_{0,cell}}{x_{1,cell} - x_{0,cell}}(r_{10}^2 - r_{00}^2) \\
 && {} + \frac{\alpha_{proj} y(\tau) - y_{0,cell}}{y_{1,cell} - y_{0,cell}}(r_{01}^2 - r_{00}^2) \\
 && {} + \left(\frac{\alpha_{proj} x(\tau) - x_{0,cell}}{x_{1,cell} - x_{0,cell}}\right)\left(\frac{\alpha_{proj} y(\tau) - y_{0,cell}}{y_{1,cell} - y_{0,cell}}\right) \\
 && {} \cdot(r_{11}^2 - r_{01}^2 - r_{10}^2 - r_{00}^2)
\end{eqnarray}





\end{document}

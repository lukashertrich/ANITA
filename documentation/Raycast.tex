\documentclass[aps,prl,twocolumn,superscriptaddress]{revtex4-1}

\usepackage{graphicx}  % this is the up-to-date package for all figures
\usepackage{amssymb}   % for math
\usepackage{verbatim}  % for the comment environment
\usepackage{color}
\usepackage[hidelinks]{hyperref}
\usepackage{listings}	% for code snippets
\usepackage{gensymb}

% Settings for code appendices
\definecolor{dkgreen}{rgb}{0,0.6,0}
\definecolor{gray}{rgb}{0.5,0.5,0.5}
\definecolor{mauve}{rgb}{0.58,0,0.82}

\lstset{frame=tb,
  language=C++,
  aboveskip=3mm,
  belowskip=3mm,
  showstringspaces=false,
  columns=flexible,
  basicstyle={\small\ttfamily},
  numbers=none,
  numberstyle=\tiny\color{gray},
  keywordstyle=\color{blue},
  commentstyle=\color{dkgreen},
  stringstyle=\color{mauve},
  breaklines=true,
  breakatwhitespace=true,
  tabsize=3
}

%\bibliographystyle{apsrev}

% these are some custom control of the page size and margins
% \topmargin= 0.2in  % these 1st two may be needed for some computers
\textheight=9in
\textwidth=6.5in
% these next two lines give us centered text
\oddsidemargin=0cm
\evensidemargin=0cm

\begin{document}

\pagestyle{headings}

\title{Polar Stereographic\\ Heightfield Intersection}

\author{Lukas R. Hertrich}
\affiliation{Student researcher at University of Hawaii at Manoa}

\begin{abstract}
In order to simulate neutrino and RF traversal of the Antarctic firn, raycasting may be used. This is accomplished by mapping bedrock elevation and ice sheet thickness
data, provided by projects such as BEDMAP2, to the WGS84 ellipsoid.
\end{abstract}

\maketitle

\section{Data Format}
Antarctic elevation and ice sheet data are provided by BEDMAP 2 in Antarctic polar stereographic projection.
The projection plane is located at 71$\degree$ south latitude and rays are cast down from the north pole to intersect the projection plane and ellipsoid.
The angle is in ellipsoidal coordinates determined by WGS84.
\section{Raycast}
The 3D raycast is parameterized with an initial position $\vec{r_0}$, a direction $\hat{d}$, and a parameter $\tau$, yielding
\begin{equation}\label{eqn:r_of_tau}
 \vec{r}(\tau) = \vec{r_0} + \tau \hat{d}
\end{equation}
This ray traverses flat, unmodified 3D cartesian space, meaning that distance traversed by the ray is equal to $\tau$.
The same ray may be parameterized in terms of normalized cartesian coordinates,
meaning that the $\vec{r_0}$ terms $r_{0x}$ and $r_{0y}$ are divided by the equatorial radius $R_{eq}$ of the WGS84 ellipsoid,
and the $r_{0z}$ term is divided by the polar radius $R_{po}$.
This yields a magnitude of radius relative to the radius of the ellipsoid in the given direction, and a ray that intersects the ellipsoid's surface
where $r_{el}(\tau) = 1$.
The magnitude of $r_{el}$ may be determined as follows:
{\tiny
\begin{equation}\label{eqn:r_el}
 r_{el}^2 = \frac{r_{0x}^2+r_{0y}^2}{R_{eq}^2}+\frac{r_{0z}^2}{R_{po}^2} \\
 + 2\tau\left(\frac{r_{0x} d_x+r_{0y} d_y}{R_{eq}^2}+\frac{r_{0z} d_z}{R_{po}^2}  \right) \\
 + \tau^2 \left( \frac{d_x^2+d_y^2}{R_{eq}^2} + \frac{d_z^2}{R_{po}^2} \right)
\end{equation}
}

\section{Heightfield Intersection}
The ray intersects the heightfield when the normalized surface radius $r_{surf}(\tau)$ is equal to the normalized ellipsoidal radius $r_{el}(\tau)$.
Their difference is then zero. Likewise the difference of their squares is zero.
\begin{equation}\label{eqn:intersection}
0 = r_{surf}(\tau) - r_{el}(\tau) 
\end{equation}\\
$r_{el}(\tau)$ may be determined by [\ref{eqn:r_el}].
$r_{surf}(\tau)$ is unity added to the quotient of the interpolated height $h(\tau)$ and the unnormalized magnitude of the ellipsoid's radius $R(\tau)$ at the given latitude.
\begin{equation}\label{eqn:r_surf}
 r_{surf}(\tau) = 1 + \frac{h(\tau)}{R(\tau)}
\end{equation}
The interpolated height $h(\tau)$ is determined by generating the vector $\vec{v}_{proj}$ from the north pole of the ellipsoid $(0, 0, R_{po})$
to the point of ray traversal $\vec{r}(\tau)$ from [\ref{eqn:r_of_tau}],
then using the point of intersection of this vector with the projection plane to calculate (u,v) coordinates with which to poll the heightfield data.
The point of intersection with the projection plane is found by determining how far the direction of $\vec{v}_{proj}$ must be traversed, in terms of its $\hat{z}$ component,
in order to hit the constant $\hat{z}$ position of the plane $z_{plane}$. This is accomplished by scaling $\vec{v}_{proj}$ by $\alpha_{proj}$ to form $\vec{v'}_{proj}$.
\begin{equation}\label{eqn:v_proj}
 \vec{v'}_{proj} = \alpha_{proj} \vec{v}_{proj}
\end{equation}
The scalar $\alpha_{proj}$ is determined as follows, with $z(\tau)$ being the $z$ component of $r(\tau)$ in [\ref{eqn:r_of_tau}]:
\begin{equation}\label{eqn:alpha_proj}
 \alpha_{proj} = \frac{-R_{po} - z_{plane}}{-R_{po} + z(\tau)}
\end{equation}
This yields the $(x'_{proj},y'_{proj})$ coordinates that may be converted to $(u,v)$ coordinates to poll and interpolate data.
\begin{equation}\label{eqn:x'_proj}
 (x'_{proj},y'_{proj}) = \alpha_{proj} (x_{proj},y_{proj})
\end{equation}
$R(\tau)$ may be determined from the geodetic latitude $\varphi$.
\begin{equation}\label{eqn:phi}
 \varphi = \arcsin{\frac{z_{el}(\tau)}{r_{el}(\tau)}}
\end{equation}

\begin{equation}\label{eqn:R_of_tau}
 R(\varphi) = \sqrt{\frac{(R_{eq}^2\cos{\varphi})^2 + (R_{po}^2\sin{\varphi})^2}{(R_{eq}\cos{\varphi})^2 + (R_{po}\sin{\varphi})^2}}
\end{equation}





\end{document}

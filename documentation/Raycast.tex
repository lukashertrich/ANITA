\documentclass[a4paper,10pt]{article}
\usepackage[utf8]{inputenc}
\usepackage{textcomp}

%opening
\title{Raycast Against Polar Stereographic Heightfield}
\author{Lukas R. Hertrich}

\begin{document}

\maketitle

%\begin{abstract}

%\end{abstract}

\section{Data Format}
Antarctic elevation and ice sheet data are provided by BEDMAP 2 in Antarctic polar stereographic projection.
The projection plane is located at 71\textdegree~ south latitude and rays are cast down from the north pole to intersect the projection plane and ellipsoid.
The angle is in ellipsoidal coordinates determined by WGS84.
\section{Raycast}
The 3D raycast is parameterized with an initial position $\vec{r_0}$, a direction $\hat{d}$, and a parameter $\tau$, yielding
\begin{equation}\label{eqn:r_of_tau}
 \vec{r}(\tau) = \vec{r_0} + \tau \hat{d}
\end{equation}
This ray traverses flat, unmodified 3D cartesian space, meaning that distance traversed by the ray is equal to $\tau$.

The same ray may be parameterized in terms of normalized cartesian coordinates,
meaning that the $\vec{r_0}$ terms $r_x$ and $r_y$ are divided by the equatorial radius $R_{eq}$ of the WGS84 ellipsoid,
and the $r_z$ term is divided by the polar radius $R_{po}$.
This yields a magnitude of radius relative to the radius of the ellipsoid in the given direction, and a ray that intersects the ellipsoid's surface
where $r_{ellipsoidal}(\tau) = 1$.
The magnitude of $r_{ellipsoidal}$ may be determined as follows:

\begin{equation}\label{eqn:r_ellipsoidal}
 r_{ellipsoidal} = \sqrt{\frac{r_x^2 + r_y^2}{R_{eq}^2} + \frac{r_z^2}{R_{po}^2}
 + 2t\left(\frac{r_x d_x + r_y d_y}{R_{eq}^2} + \frac{r_z d_z	}{R_{po}^2}  \right)
 + t^2 \left( \frac{d_x^2 + d_y^2}{R_{eq}^2} + \frac{d_z^2}{R_{po}^2} \right)}
\end{equation}

\section{Intersection With Heightfield}
The ray intersects the heightfield when the normalized height radius $r_{height}(\tau)$ is equal to the normalized ellipsoidal radius $r_{ellipsoidal}(\tau)$.
Their difference is then zero. Likewise the difference of their squares is zero.

\begin{equation}\label{eqn:intersection}
0 = r_{height}(\tau) - r_{ellipsoidal}(\tau) 
\end{equation}

$r_{ellipsoidal}(\tau)$ may be determined by [\ref{eqn:r_ellipsoidal}].
$r_{height}(\tau)$ is the quotient of the interpolated height $h(\tau)$ and the unnormalized magnitude of the ellipsoid's radius $R(\tau)$ in the given direction.

\begin{equation}\label{eqn:r_height}
 r_{height}(\tau) = \frac{h(\tau)}{R(\tau)}
\end{equation}

The interpolated height $h(\tau)$ is determined by generating the vector $\vec{v}_{proj}$ from the north pole of the ellipsoid $(0, 0, R_{po})$
to the point of ray traversal $\vec{r}(\tau)$ (given by [\ref{eqn:r_of_tau}]),
then using the point of intersection of this vector with the projection plane to calculate (u,v) coordinates with which to poll the heightfield data.
The point of intersection with the projection plane is found by determining how far $\vec{v}_{proj}$ must be traversed, in terms of its $\hat{z}$ component,
in order to hit the constant $\hat{z}$ position of the plane.

\end{document}
